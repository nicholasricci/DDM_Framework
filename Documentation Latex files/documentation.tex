\documentclass[a4paper,11pt]{report}

\usepackage[latin1]{inputenc}
\usepackage[T1]{fontenc}
\usepackage[italian]{babel}
\usepackage{verbatim}
\usepackage{graphicx}
\usepackage{listings}
\usepackage{xstring}
\usepackage{color}

\title{Documentation Framework DDM}
\author{Nicholas Ricci}
\date{25/09/2014}

%format images
\newcommand{\customfigure}[3]{
  \begin{figure}[ht]
    \centering
    \includegraphics[scale=0.35, width=125mm]{#1}
    \caption{#2}
    \label{fig:#3}
  \end{figure}
}

%definition of some color
\definecolor{foldercolor}{RGB}{120, 120, 120}
\definecolor{commandcolor}{RGB}{255, 128, 0}
\definecolor{sourcecolor}{RGB}{96, 206, 181}
\definecolor{headercolor}{RGB}{249, 126, 179}
\definecolor{bashvariable}{RGB}{0, 102, 0}
\definecolor{bashfilecolor}{RGB}{0, 0, 0}

%Do capital letter in bold
\newcommand{\majorword}[1]{
  \textbf{\StrLeft{#1}{1}}\StrGobbleLeft{#1}{1}
}

%Do format for folders
\newcommand{\folder}[1]{
  \textcolor{foldercolor}{\textit{#1}}
}

%Do format for commands
\newcommand{\command}[1]{
  \textcolor{commandcolor}{\textit{#1}}
}

%Do format for sources
\newcommand{\source}[1]{
  \textcolor{sourcecolor}{\textit{#1}}
}

%Do format for headers
\newcommand{\header}[1]{
  \textcolor{headercolor}{\textit{#1}}
}

%Do format for c variable
\newcommand{\variable}[1]{
  \textcolor{blue}{\textit{#1}}
}

%Do format for bash variable
\newcommand{\bashvariable}[1]{
  \textcolor{bashvariable}{\textit{#1}}
}

%Do format for bash variable
\newcommand{\bashfile}[1]{
  \textcolor{bashfilecolor}{\textbf{#1}}
}

\lstset{language=C,
        basicstyle=\ttfamily\scriptsize,
        keywordstyle=\color{blue}\ttfamily,
        stringstyle=\color{red}\ttfamily,
        commentstyle=\color{green}\ttfamily,
        morecomment=[l][\color{magenta}]{\#},
        morekeywords={DDM_Timer},
        frame={singleline}
}

\begin{document}
 \maketitle
 \tableofcontents
 
 \chapter{Introduction}
 
 This is a framework for \majorword{Data} \majorword{Distribution} \majorword{Management}.
 Which this framework you can put your algorithms inside and you can test them with other algorithms with coherent
 input data.
 
 In this document I'll explain how to configure this framework and what to do to edit your algorithms to make sure
 your algorithms works fine inside of it.
 
 Before start you must be sure that your algorithms contain a makefile. If you haven't, do it!
 
 \chapter{Fix your algorithm}
 
 Before you can start to use framework you must fix your algorithm.
 Inside the root framework there's a folder called \folder{C libraries for this framework}, within it there are
 two files: \source{DDM\_input\_output.c} and \header{DDM\_input\_output.h}.
 
 You need to copy this two files into your sources folder.
 Remember to edit your makefile to compile also this library and to update your clean target.
 
 Now you need to open your main file and include this library (for example):
 \begin{lstlisting}
  #include "DDM_input_output.h"
 \end{lstlisting}
 
 Another thing you must do is to declare a new \variable{DDM\_Timer}:
 \begin{lstlisting}
  DDM_Timer ddm_timer;
 \end{lstlisting}

 Now you can get the number of extents, updates, subscriptions, alfa, dimensions:
 \begin{lstlisting}
  size_t extents = DDM_Get_Extents(argc, argv);
  size_t dimension = DDM_Get_Dimension(argc, argv);
  size_t updates = DDM_Get_Updates(argc, argv);
  size_t subscriptions = DDM_Get_Subscriptions(argc, argv);
  float alfa = DDM_Get_Alfa(argc, argv);
 \end{lstlisting}

 After you've gotten parameters it's time to start the timer, execute the matching algorithm and
 at the end stop the timer:
 \begin{lstlisting}
  DDM_Start_Timer(&ddm_timer);
    //execute the matching algorithm
  DDM_Stop_Timer(&ddm_timer);
 \end{lstlisting}

  At the end you need to write result into a file then:
  \begin{lstlisting}
   DDM_Write_Result(argv, DDM_Get_Total_Time(ddm_timer));
  \end{lstlisting}
  
  After you've fix your algorithm you can put it inside the framework and go to the next chapter!
 
 \chapter{How to use launcher}
 
 The script in bash inside of the root, launcher.sh, is a launcher to execute the framework.
 To execute the script you can open your terminal and digit this:
 \begin{lstlisting}[language=bash]
  ./launcher.sh <command>
 \end{lstlisting}

 Below will be indicated commands in order to execute for a proper functioning of the framework:
 \begin{enumerate}
  \item \command{--help}: will be shown what the framework can do and which commands can be run;
  \item \command{configure}: will run an interactive part in which the user must configure the following 
  parameters:
  \begin{itemize}
   \item \bashvariable{START\_EXTENTS}: indicating the initial number of extents by which the 
   framework will execute the test;
   \item \bashvariable{MAX\_EXTENTS}: indicating the maximum number of extents to which the 
   framework will come to run tests;
   \item \bashvariable{STEP\_SIZE}: indicating the number by which the framework will increase 
   the number of initial extents until reaches the maximum;
   \item \bashvariable{DIMENSION}: indicating the number of dimensions with which the framework 
   will execute the algorithms;
   \item \bashvariable{CORES}: in the case of parallelization in OpenMP this parameter must 
   be set with the maximum number of cores that the user wants to use as the maximum number of cores. 
   The initial number of cores is set to 2 by default;
   \item \bashvariable{ALFAS}: is an array that indicates the degree of overlap of subscription extents 
   and updates extents. You can insert multiple elements;
   \item \bashvariable{ALFAS\_PAR}: is an array similar to ALFAS, the only difference is that 
   this is to indicate the values to use with algorithms that use OpenMP;
   \item \bashvariable{RUN}: indicating the number of executions that you want to perform for each algorithm.
  \end{itemize}
  This command will create a file in the root of the framework called \bashfile{configure.sh};
  \item \command{build}: will be compiled all algorithms in the folder \folder{Algorithms}.
  At the end of the process there will be two text files in the root of each algorithm:
  \begin{itemize}
   \item DDM\_Sequential: in this file is necessary to write all names of sequential version 
   of executables. One for each line;
   \item DDM\_Parallel: in this file is necessary to write all names of parallel version of
   executables. One for each line.
  \end{itemize}
  Enter names of executables is a critical step, and even more fundamental is to insert only 
  one name on each line;
  \item \command{run}:
  \begin{itemize}
   \item \folder{\_results}: in this folder will be available all the runs performed with 
   execution time;
   \item \folder{\_graphs}: in this folder will be available files with the number 
   of extents on the left and to the right will be the average time of the runs performed.
   \end{itemize}
 \end{enumerate}

 Questi comandi sono stati elencati in base al loro ordine di esecuzione.

 Sono presenti altri tre comandi utili per tenere pulito il framework:
 \begin{itemize}
  \item \command{cleanalgorithms}: for each algorithm executes command \command{make clean}, 
  then is foundamental define a clean target in your makefile;
  \item \command{cleanresults}: will delete \folder{\_results} and \folder{\_graphs} folders 
  with all their files;
  \item \command{clean}: executes both cleanalgorithms and cleanresults.
 \end{itemize}
 
\end{document}

